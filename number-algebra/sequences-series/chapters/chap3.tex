\chapter{Geometric Sequences and Series}

\section{General Formula}

\begin{proposition}
	  The general formula for the $n$th term of a geometric sequence is given by:
  \begin{equation}
	u_n = u_1 \times r^{n-1}
  \end{equation}
  where $u_1$ is the first term and $r$ is the common ratio.
\end{proposition}
\begin{proof}
	We know that the first term is simply $u_1$. We obtain the second term by multiplying the first term by $r$:
	\begin{equation}
		u_2 = u_1 \times r
	\end{equation}
	We employ a similar approach for the remaining terms:
	\begin{align}
		u_3 &= u_2 \times r = u_1 \times r^2 \\
		u_4 &= u_3 \times r = u_1 \times r^3 \\
		&\vdots \\
		u_n &= u_1 \times r^{n-1}
	\end{align}
	This proof is quite similar to that of the arithmetic sequence's general formula.
\end{proof}

As with arithmetic sequences, solving problems relating to geometric sequences
requires you to identify the first term and the common ratio. From there, you can 
plug values into the general formula to find the desired term.

\section{Geometric Series}

\begin{definition}[Geometric Series]
	A \textit{geometric series} is the sum of the terms of a geometric sequence. If we take the first $n$ terms of a geometric sequence
	and add them together, we get a finite geometric series. 
\end{definition}

With geometric sequences, there is an additional consideration. The sums of infinite geometric 
series are not necessarily undefined, as with arithmetic series. 

\begin{definition}[Convergence]
	A series is said to \textit{converge} if the sum approaches a finite value as more terms are added. 
	If the sum does not approach a finite value, the series is said to \textit{diverge}. To determine 
	whether a geometric series converges or diverges, we look at the common ratio, $r$. If $|r| < 1$, the series converges; 
	otherwise, it diverges.
\end{definition}

\begin{intuition}
	To understand why a geometric series converges when $|r| < 1$, consider the behavior of the terms in the series. 
	When the common ratio $r$ is between -1 and 1, each successive term in the geometric sequence becomes smaller in magnitude. 
	As more terms are added to the series, the contributions of these smaller terms become negligible, leading the sum to approach a finite limit.
	On the other hand, if $|r| \geq 1$, the terms do not decrease in magnitude, and the sum continues to grow without bound, resulting in divergence.
\end{intuition}

We will consider the above in more detail when we consider the infinite sum of a geometric
series. First, let's start with finite sums:

\begin{proposition}
	The sum of the first $n$ terms of a geometric series can be calculated using the formula:
	\begin{equation}
		S_n = u_1 \frac{1 - r^n}{1 - r} \quad \text{for } r \neq 1
	\end{equation}
	where $u_1$ is the first term and $r$ is the common ratio.
\end{proposition}
\begin{proof}
	Let the first $n$ terms of a geometric series be denoted by $S_n$:
	\begin{align}
		S_n &= u_1 + u_2 + u_3 + \ldots + u_n \\
		&= u_1 + u_1 r + u_1 r^2 + \ldots + u_1 r^{n-1}\label{eq:proof-3-1}
	\end{align}
	Multiplying both sides by $r$ gives:
	\begin{equation}\label{eq:proof-3-2}
		r S_n = u_1 r + u_1 r^2 + u_1 r^3 + \ldots + u_1 r^n
	\end{equation}
	Subtracting equation \eqref{eq:proof-3-2} from equation \eqref{eq:proof-3-1}, we get:
	\begin{align*}
		S_n - r S_n &= u_1 - u_1 r^n \\
		S_n (1 - r) &= u_1 (1 - r^n) \\
		S_n &= u_1 \frac{1 - r^n}{1 - r}
	\end{align*}
\end{proof}

The formula is valid for all geometric series. We employ a similar proof for the sum 
of an infinite geometric series, which only applies when the series converges.

\begin{proposition}
	The sum of an infinite geometric series converges to:
	\begin{equation}
		S_\infty = \frac{u_1}{1 - r} \quad \text{for } |r| < 1
	\end{equation}
	where $u_1$ is the first term and $r$ is the common ratio.
\end{proposition}
\begin{proof}
	Let's start at the same place as we did for the finite sum, by writing out th complete sum:
	\begin{equation}\label{eq:proof-3-3}
		S_\infty = u_1 + u_1 r + u_1 r^2 + u_1 r^3 + \ldots
	\end{equation}
	Multiplying both sides by $r$ gives:
	\begin{equation}\label{eq:proof-3-4}
		r S_\infty = u_1 r + u_1 r^2 + u_1 r^3 + \ldots
	\end{equation}
	Subtracting equation \eqref{eq:proof-3-4} from equation \eqref{eq:proof-3-3}, we get:
	\begin{align*}
		S_\infty - r S_\infty &= u_1 \\
		S_\infty (1 - r) &= u_1 \\
		S_\infty &= \frac{u_1}{1 - r}
	\end{align*}
	However, this only holds true if the series converges, which occurs when $|r| < 1$.
\end{proof}

It is not uncommon for questions on geometric series to feature polynomial functions as terms 
rather than simple numbers. The approach to solving such problems remains the same; you must identify 
the first term and the common ratio, then apply the relevant formula.

\begin{exercise}[Hard]
	The first three terms of a gemetric sequence are $x+4$, $6x$ and $2x^2$. Find:
	\begin{enumerate}[label=\alph*)]
		\item the fifth term, $u_5$
		\item the sum of the first seven terms, $S_7$
	\end{enumerate}
\end{exercise}
\begin{answer}
	In this question, we are given the first term of the sequence for free. Hence, 
	all we need to do is find the common ratio. While this may seem somewhat complicated,
	given that the terms are polynomials, the process is exactly the same as with numerical
	terms. The common ratio of consecutive terms is given by:
	\begin{equation*}
		r = \frac{u_{n+1}}{u_n}
	\end{equation*}
	First, we substitute $n=1$ to find the common ratio:
	\begin{align}
		r &= \frac{u_2}{u_1} \\
		&= \frac{6x}{x + 4} \label{eq:geo-ex-1}
	\end{align}
	We can also substitute $n=2$ to find the common ratio:
	\begin{align}
		r &= \frac{u_3}{u_2} \\
		&= \frac{2x^2}{6x} \\
		&= \frac{x}{3} \label{eq:geo-ex-2}
	\end{align}
	Since this is a geometric sequence, the ratio between consecutive terms must always be 
	the same, so the results in equations \eqref{eq:geo-ex-1} and \eqref{eq:geo-ex-2} must be equal:
	\begin{align*}
		\frac{6x}{x + 4} &= \frac{x}{3} \\
		18x &= x^2 + 4x \\
		x^2 - 14x &= 0 \\
		x(x - 14) &= 0
	\end{align*}
	Hence, $x=0$ or $x=14$. If $x=0$, all the terms of the sequence (other than the first) will also 
	be $0$ (each consecutive term is multiplied by $0$). This is a trivial case, so it's not very interesting. 
	It is far more likely that the question is asking for the non-trivial case, so we take $x=14$.
	Substituting $x=14$ into equation \eqref{eq:geo-ex-2}, we find the common ratio:
	\begin{equation*}
		r = \frac{x}{3} = \frac{14}{3}
	\end{equation*}
	\begin{enumerate}[label=\alph*)]
		\item Now that we have both the first term and the common ratio, we can find the fifth term using the general formula:
		\begin{align*}
			u_5 &= u_1 \times r^{5-1} \\
			&= (14 + 4) \times \left(\frac{14}{3}\right)^4 \\
			&= 18 \times \frac{38416}{81} \\
			&= \frac{691,488}{81}
		\end{align*}
		\item Next, we find the sum of the first seven terms using the sum formula for a geometric series:
		\begin{align*}
			S_7 &= u_1 \frac{1 - r^7}{1 - r} \\
			&= (14 + 4) \frac{1 - \left(\frac{14}{3}\right)^7}{1 - \frac{14}{3}} \\
			&= 18 \frac{1 - \frac{105,413,504}{2187}}{\frac{-11}{3}} \\
			&= 18 \times \frac{-3}{11} \times \left(1 - \frac{105,413,504}{2187}\right) \\
			&= \frac{-54}{11} \times \frac{-105,411,317}{2187} \\
			&= \frac{5,693,008,518}{24,057}
		\end{align*}
	\end{enumerate}
\end{answer}

\begin{exercises}[Geometric Sequences and Series]
\end{exercises}
\begin{questions}
	% Put multicols outside enumerate so list indentation is computed per column
	\begin{multicols}{2}
		\begin{enumerate}[label=\alph*), itemsep=1ex]
			\item Consider the geometric sequence where the first term is $4$ and the common ratio is $3$. Find:
			\begin{enumerate}[label=\roman*)]
				\item The first five terms of the sequence.
				\item The general formula for the $n$th term.
				\item The sum of the first 6 terms.
			\end{enumerate}

			\item An geometric sequence has $u_3 = 72$ and $u_{6} = 2592$. Find:
			\begin{enumerate}[label=\roman*)]
				\item The first term and common ratio.
				\item The general formula for the $n$th term.
				\item The sum of the first 10 terms.
			\end{enumerate}
		\end{enumerate}
	\end{multicols}
	\begin{multicols}{2}
		\begin{enumerate}[label=\alph*), itemsep=1ex, start=3]
			\item A certain bacteria population doubles every 4 hours. If the initial population is 500 bacteria at time $t=0$, find:
			\begin{enumerate}[label=\roman*)]
				\item The population after 24 hours.
				\item The general formula for the population after $t$ hours.
				\item The total population growth over the first 24 hours.
			\end{enumerate}

			\item The first three terms of a geometric sequence are $2p+3$, $3$, and $p-2$. 
			\begin{enumerate}[label=\roman*)]
				\item Show that $p$ satisfies the equation $2p^2 - p - 15 = 0$.
				\item Given that the sequence has an infinite sum, find the values of $p$ and $r$.
				\item Find the sum of all terms of the sequence.
			\end{enumerate}
		\end{enumerate}
	\end{multicols}
	\vspace*{0.2cm}
\end{questions}