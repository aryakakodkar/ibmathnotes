\chapter{Induction}
\section{Proofs by Induction}

Proofs by induction are a powerful method of proving statements 
that are asserted to be true for all natural numbers (or some well-defined subset
of them). We remind ourselves:

\begin{definition}[Natural Numbers]\label{def:natural-numbers}
    The set of positive integers: $\{1, 2, 3, \ldots\}$.
\end{definition}

Sometimes, \hyperref[def:natural-numbers]{the natural numbers} are defined to include 0,
but we will not consider this definition here. A proof by induction is defined as follows:

\begin{definition}[Proof by Induction]\label{def:proof-by-induction}
    A \textit{proof by induction} is a method of proving a statement by showing that if it holds for a natural number $n$, then it also holds for $n+1$. The proof consists of two main steps:
    \begin{enumerate}
        \item \textbf{Base Case:} Prove that the statement holds for the initial value (usually $n=1$).
        \item \textbf{Inductive Step:} Assume the statement holds for some arbitrary natural number $n=k$, and then prove it holds for $n=k+1$.
    \end{enumerate}
\end{definition}

\hyperref[def:proof-by-induction]{Proofs by induction} are far more intuitive than their definition suggests. Let us
first consider the concept of an inductive step: if we can show that the statement holds for $n=k+1$,
assuming it holds for $n=k$, then we can conclude that the statement holds for all natural numbers greater than or equal to $k$.
We could define another variable, $m$,
\begin{equation*}
    m = k + 1
\end{equation*}

We know that if the statement holds for $n=k$, then it must also hold for $n=m$. But then, we can apply the same logic to $m$:
if the statement holds for $n=m$, then it must also hold for $n=m+1$. Of course:
\begin{equation*}
    m + 1 = (k + 1) + 1 = k + 2
\end{equation*}
Therefore, if the statement holds for $n=k$, it must also hold for $n=k+2$. We can repeat this process indefinitely, showing that
if the statement holds for $n=k$, it must also hold for all natural numbers greater than $k$.

The keen observer will note that we made an assumption at the beginning of this process: we assumed 
that the statement holds for $n=k$. What ensures that this assumption is valid? This is where the base case comes in.
By proving the base case (that the statement holds for $n=1$), we establish a starting point for our inductive step.
From the base case, we know that the statement holds for $n=1$. Therefore, by the inductive step, it must also hold for $n=1+1=2$.
Then, by applying the inductive step again, it must hold for $n=3$, and so on. Thus, by proving both the base case and the inductive step,
we can conclude that the statement holds for all natural numbers. As always, an example is better than an explanation:

% TODO: add this to the direct proofs as well
\begin{exercise}
    Prove by mathematical induction that the sum of the first $n$ natural numbers is given by the formula:
    \begin{equation}\label{eq:inductive-proof-ex1}
        S(n) = \frac{n(n+1)}{2}
    \end{equation}
    where $S(n)$ is the sum of the first $n$ natural numbers.
\end{exercise}
\begin{answer}
    A quick note: this is the same problem that we saw in the \hyperref[sec:proofs]{direct proofs} section. Almost always, there is more than 
    one way to prove any statement. Let's proceed with the proof by induction:
    \newline\newline
    \textbf{Base Case:} For our base case, we must prove that the statement holds for the smallest 
    value of $n$. In this case, the statement concerns the natural numbers, so the smallest value of $n$ is 1. 
    We check that equation \ref{eq:inductive-proof-ex1} holds for $n=1$:
    \begin{equation*}
        S(1) = \frac{1(1+1)}{2} = 1
    \end{equation*}
    The sum of the first 1 natural number is indeed 1, so the base case holds.
    \newline\newline
    \textbf{Inductive Step:} For the inductive step, we assume that the statement holds for some arbitrary natural number $n=k$. 
    This is our inductive hypothesis. We must then prove that the statement holds for $n=k+1$. 
    According to our inductive hypothesis, we have:
    \begin{equation}\label{eq:inductive-proof-ex1-hyp}
        S(k) = \frac{k(k+1)}{2}
    \end{equation}
    We must show that:
    \begin{equation*}
        S(k+1) = \frac{(k+1)(k+2)}{2}
    \end{equation*}
    The sum of the first $k+1$ natural numbers is simply the sum of the first $k$ natural numbers plus $k+1$:
    \begin{align*}
        S(k+1) &= S(k) + (k+1) 
    \end{align*}
    From our inductive hypothesis (equation \ref{eq:inductive-proof-ex1-hyp}), we can substitute for $S(k)$:
    \begin{align*}
                S(k+1) &= \frac{k(k+1)}{2} + (k+1) \\
                &= \frac{k(k+1) + 2(k+1)}{2}  \\
                &= \frac{k^2 + k + 2k + 2}{2} \\
                &= \frac{k^2 + 3k + 2}{2} \\
                &= \frac{(k+1)(k + 2)}{2}
    \end{align*}
    We see that the last line of this equation can be rewritten to match equation \ref{eq:inductive-proof-ex1}:
    \begin{equation*}
        S(k+1) = \frac{(k+1)\left([k + 1] + 1\right)}{2}
    \end{equation*}
    Thus, we have shown that if the statement holds for $n=k$, it also holds for $n=k+1$.
    \newline\newline
    By the principle of mathematical induction, since we have proven both the base case and the inductive step,
    we conclude that the formula for the sum of the first $n$ natural numbers holds for all natural numbers $n$.
    Since it holds for $n=1$, it holds for $n=2$, and so on, for all natural numbers.
\end{answer}

\begin{exercise}
    Prove by mathematical induction that $11^n - 6$ is divisible by 5 for all natural numbers $n \geq 1$.
\end{exercise}
\begin{answer}
    We will prove this statement using mathematical induction.
    \newline\newline
    \textbf{Base Case:} For our base case, we need to verify that the statement holds for $n=1$:
    \begin{equation*}
        11^1 - 6 = 11 - 6 = 5
    \end{equation*}
    Since 5 is divisible by 5, the base case holds.
    \newline\newline
    \textbf{Inductive Step:} For the inductive step, we assume that the statement holds for some arbitrary natural number $n=k$. 
    This is our inductive hypothesis. We must then prove that the statement holds for $n=k+1$. 
    According to our inductive hypothesis, we have:
    \begin{equation*}
        11^k - 6 \text{ is divisible by } 5
    \end{equation*}
    This means there exists an integer $m$ such that:
    \begin{equation}\label{eq:inductive-proof-ex2-hyp}
        11^k - 6 = 5m
    \end{equation}
    We need to show that:
    \begin{equation*}
        11^{k+1} - 6 \text{ is divisible by } 5
    \end{equation*}
    We can express $11^{k+1}$ as $11^k \cdot 11$:
    \begin{align*}
        11^{k+1} - 6 &= 11^k \cdot 11 - 6 \\
                     &= 11(11^k) - 6
    \end{align*}
    Now, we can rewrite this expression using our inductive hypothesis (equation \ref{eq:inductive-proof-ex2-hyp}):
    \begin{align*}
        11^{k+1} - 6 &= 11(11^k) - 6 \\
                     &= 11(5m + 6) - 6\\
                     &= 55m + 66 - 6 \\
                     &= 55m + 60 \\
                     &= 5(11m + 12)
    \end{align*}
    Since $11m + 12$ is an integer, we have shown that $11^{k+1} - 6$ is divisible by 5.
    \newline\newline
    By the principle of mathematical induction, since we have proven both the base case and the inductive step,
    we conclude that $11^n - 6$ is divisible by 5 for all natural numbers $n \geq 1$.
\end{answer}