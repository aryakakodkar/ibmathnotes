\chapter{Introduction}
\section{Definitions}

\begin{definition}[Sequence]\label{def:sequence}
  A \textit{sequence} is an ordered list of numbers which follows
a specific pattern. Each number in this pattern is called a \textit{term}. Sequences
can be finite or infinite (i.e. they may or may not have a last term).
\end{definition}

\begin{eg}
  Some examples of sequences include:
  \begin{itemize}
    \item $2, 4, 6, 8, 10, \ldots$ (even numbers)
    \item $1, 4, 9, 16, 25, \ldots$ (perfect squares)
    \item $1, \frac{1}{2}, \frac{1}{3}, \frac{1}{4}, \ldots$ (reciprocals of natural numbers)
  \end{itemize}
  \vskip0.5cm
\end{eg}

\vskip0.2cm
For the rest of these definitions, we'll use the first of these examples. Note that when
a \hyperref[def:sequence]{sequence} is written with ellipses ($\ldots$), it is assumed to be infinite. The first term
of this sequence is $2$, the second is $4$, the third is $6$, and so on.
But what about the \nth{100} term? Clearly we can't just keep counting! Instead, we define a \textit{general formula} for the $n$th term,
which allows us to quickly calculate the value of any term.

\begin{notation}
  The $n$th term of a sequence is denoted by $u_n$. In this case, the first term is $u_1=2$, the 
  second is $u_2=4$, and so on.
\end{notation}

The formula for the $n$th term (for the sequence of even numbers) can then be written as:
\begin{equation}
  u_n = 2n
\end{equation}

To find the value of any term in the sequence, we simply substitute the term number, $n$, into the general formula.
Now, we can easily find the \nth{100} term:
\begin{equation*}
  u_{100} = 2 \times 100 = 200
\end{equation*}

Clearly, this is much more efficient than counting all the way up to the \nth{100} term!

\begin{definition}[Series]\label{def:series}
  A \textit{series} is the sum of the terms of a sequence. If we take the first $n$ terms of a sequence
  and add them together, we get a finite series. If we add up all the terms of an infinite sequence,
  we get an infinite series.
\end{definition}

For the \hyperref[def:sequence]{sequence} we defined earlier ($2, 4, 6, 8, \ldots$), the \hyperref[def:series]{series} formed by adding the
first $n$ terms is denoted by $S_n$. In this case:

\begin{align*}
  S_1 &= 2 \\
  S_2 &= 2 + 4 = 6 \\
  S_3 &= 2 + 4 + 6 = 12 \\
  \vdots \\
  S_n &= 2 + 4 + 6 + \ldots + 2n \\
  \vdots \\
  S_\infty &= 2 + 4 + 6 + 8 + \ldots
\end{align*}

Where $S_\infty$ represents the infinite \hyperref[def:series]{series} formed by adding all the terms of the \hyperref[def:sequence]{sequence}. Note that
this infinite series `diverges,' meaning that the terms of the sum keep growing, so the sum's value is undefined.
We will discuss this in more detail later on.

\section{General Formulae}

Finding general formulae is a key part of working with \hyperref[def:sequence]{sequences} and \hyperref[def:series]{series}. Often,
the challenge lies in identifying the pattern. In this section,
I'll discuss some common types of sequences, and hopefully demonstate what to look out for.

\begin{definition}[Arithmetic Sequence]\label{def:arithmetic_sequence}
  An \textit{arithmetic sequence} is a sequence in which the difference between consecutive terms is a constant, called the common difference, $d$.
\end{definition}

\begin{exercise}
  Find the general formula for the following arithmetic sequence:
  \begin{equation*}
    5, 11, 17, 23, 29, \ldots
  \end{equation*}
\end{exercise}
\begin{answer}
  You can quickly identify an \hyperref[def:arithmetic_sequence]{arithmetic sequence} by checking whether the difference between consecutive terms
  remains constant. In this case, the common difference is $d=6$. The general formula for the $n$th term can be expressed as:
  \begin{equation*}
    u_n = u_1 + (n-1)d = 5 + (n-1) \times 6 = 6n - 1
  \end{equation*}

  \noindent We'll discuss where this formula comes from in the next section.
\end{answer}


\begin{definition}[Geometric Sequence]\label{def:geometric_sequence}
  A \textit{geometric sequence} is a sequence in which each term is found by multiplying the previous term by a constant, called the common ratio, $r$.
\end{definition}

\begin{exercise}
  Find the general formula for the following geometric sequence:
  \begin{equation*}
    3, 6, 12, 24, 48, \ldots
  \end{equation*}
\end{exercise}
\begin{answer}
  Recognizing a \hyperref[def:geometric_sequence]{geometric sequence} is similarly straightforward; check whether the
  ratio (division) between consecutive terms remains constant. In this case, the common ratio is $r=2$. The general formula for the $n$th term can be expressed as:
  \begin{equation*}
    u_n = u_1 \times r^{n-1} = 3 \times 2^{n-1}
  \end{equation*}

  \noindent We'll discuss where this formula comes from in the next section.
\end{answer}

There are a few other common sequences which you should recognize on sight, such as:

\begin{itemize}
  \item Perfect squares: $1, 4, 9, 16, 25, \ldots$ with general formula $u_n = n^2$
  \item Perfect cubes: $1, 8, 27, 64, 125, \ldots$ with general formula $u_n = n^3$
\end{itemize}

It is also possible that you may be given a sequence of fractions, where either the numerator,
denominator, or both follow a specific pattern. In such cases, we can consider the patterns separately.

\begin{exercise}
  Find the general formula for the following sequence:
  \begin{equation*}
    \frac{1}{2}, \frac{3}{4}, \frac{5}{8}, \frac{7}{16}, \ldots
  \end{equation*}
\end{exercise}
\begin{answer}
  In this case, the numerators follow an \hyperref[def:arithmetic_sequence]{arithmetic sequence} with a common difference of $2$, 
  while the denominators follow a \hyperref[def:geometric_sequence]{geometric sequence} with a common ratio of $2$. 
  Thus, we can express the general term as:
  \begin{equation*}
    u_n = \frac{1 + 2(n-1)}{2\times 2^{n - 1}}
  \end{equation*}
\end{answer}

\section{Sigma Notation}\label{sec:sigma-notation}

When dealing with \hyperref[def:series]{series}, especially those with many terms, it 
is often convenient to use sigma notation to represent the sum concisely. Here is an example of 
a sum expressed in sigma notation:
\begin{eg}
  The sum of the first $n$ even numbers can be expressed as:
  \begin{equation*}
    S_n = \sum_{i=1}^{n} 2i
  \end{equation*}
\end{eg}
\begin{explanation}
  In the above example, the statement beneath the sigma symbol ($\sum$) indicates that we are initializing a variable $i=1$.
  We then increment the variable until it reaches $n$. This maximum value is indicated
  by the number (or variable) above the sigma symbol. The terms of the sum are given by the expression inside the sigma symbol,
  telling us to sum $2i$ for each value of $i$ from $1$ to $n$. Hence, the above example tell us:
  \begin{equation*}
    S_n = 2(1) + 2(2) + 2(3) + \ldots + 2(n)
  \end{equation*}
\end{explanation}

As we said earlier, this sum diverges, so it doesn't have a definite value. If we instead consider the sum of the 
reciprocals of the even numbers, we find that it actually does have a value:
\begin{equation*}
  S_\infty = \sum_{i=1}^{\infty} \frac{1}{2i} = \frac{1}{2} + \frac{1}{4} + \frac{1}{6} + \ldots = 1
\end{equation*}

It's quite strange that the sum of an infinite number of terms can have a finite value. We'll explore this more 
with a geometric proof later on.

\begin{exercises}[Introduction to Sequences and Series]
\end{exercises}
\begin{questions}
  \begin{multicols}{2}
    \begin{enumerate}[label = \alph*), wide, leftmargin = *, itemsep = 1ex, after = \setcounter{enumi}{0}]
      \item Find the general formula for the following sequences:
      \begin{tasks}[label=\roman*)](1)
        \task $1, 3, 5, 7, 9, \ldots$
        \task $2, 6, 18, 54, 162, \ldots$
        \task $1, \frac{1}{4}, \frac{1}{9}, \frac{1}{16}, \ldots$
        \task $4, 9, 16, 25, 36, \ldots$
      \end{tasks}
      \vfill\null
      \columnbreak
      \item Task B
      \begin{tasks}(2)[label=\roman*]
      \task Item b1)
      \task Item b2)
      \task Item b3)
      \task Item b4)
      \end{tasks}
          \item Task C \begin{tasks}(2)
      \task Item b1)
      \task Item b2)
      \task Item b3)
      \task Item b4)
      \end{tasks}
    \end{enumerate}
  \end{multicols}
\end{questions}