\chapter{Applications of Sequences and Series}

\section{Simple Interest}

\begin{definition}[Interest]
  \textit{Interest} is the cost of borrowing money, usually expressed as a percentage of the principal amount
  borrowed, over a specific period of time.
\end{definition}

\begin{definition}[Simple Interest]
  \textit{Simple interest} is a way of calculating interest on a principal amount
  where the interest is calculated only on the original principal, not on the accumulated interest.
\end{definition}

If you're not familiar with either of the above defitions (or both), I hope that 
the following example will help clarify things. 

\begin{eg}
	Let's say you borrow \$1000 from a bank at an annual simple interest rate of 5\% for 3 years. 
	The \$1000 your borrowed is called the \textit{principal} amount. The interest rate is 5\% per year, which means that
	each year, you will owe 5\% of the principal amount as interest. That means that your annual interest (payment) is:
	\begin{equation*}
		\text{Annual Interest} = \text{Principal} \times \text{Interest Rate} = 1000 \times 0.05 = \$50
	\end{equation*}
	So, each year, you owe \$50 in interest. Over 3 years, the total interest you owe is:
	\begin{equation*}
		\text{Total Interest} = \text{Annual Interest} \times \text{Number of Years} = 50 \times 3 = \$150
	\end{equation*}
	If, instead of being asked to pay the interest annually, you are asked to pay the interest (and the principal) back 
	all at once at the end of the 3 years, you would owe:
	\begin{align*}
		\text{Principal} + (\text{Annual Interest} \times \text{Number of Years}) &= 1000 + (50 \times 3) = \$1150
	\end{align*}
	Note that it's fairly straightforward to extend this example to any number of years. For $n$ years worth of interst, we find:
	\begin{align*}
		\text{Total Interest} &= \text{Annual Interest} \times \text{Number of Years} = 50 \times n \\
		\text{Total Amount Owed} &= \text{Principal} + (\text{Annual Interest} \times \text{Number of Years}) = 1000 + (50 \times n)
	\end{align*}
	Hopefully, you can see that the total amount owed forms an arithmetic sequence, with the $n$th term corresponding to the total amount owed
	after $n$ years. Each year, the amount increases by a constant amount (\$50), which is the common difference of the arithmetic sequence.
\end{eg}

It's important to note that in the above case of simple interest, the interest payment remains
constant each year because it's calculated only based on the original principal. This is in contrast to compound interest. 

\section{Compound Interest}

\begin{definition}
	  \textit{Compound interest} is a way of calculating interest where the interest is calculated on both the original principal
  and the accumulated interest from previous periods.
\end{definition}

Essentially, with compound interest, you generate interest, which is added back onto the principal. 
Consequently, you earn interest on the new, larger principal amount in the next period. 
This leads to an exponential growth of the total amount owed, as opposed to linear growth. 