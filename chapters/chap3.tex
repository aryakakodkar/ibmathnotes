\chapter{Geometric Sequences and Series}

\section{General Formula}

\begin{proposition}
	  The general formula for the $n$th term of a geometric sequence is given by:
  \begin{equation}
	u_n = u_1 \times r^{n-1}
  \end{equation}
  where $u_1$ is the first term and $r$ is the common ratio.
\end{proposition}
\begin{proof}
	We know that the first term is simply $u_1$. We obtain the second term by multiplying the first term by $r$:
	\begin{equation}
		u_2 = u_1 \times r
	\end{equation}
	We employ a similar approach for the remaining terms:
	\begin{align}
		u_3 &= u_2 \times r = u_1 \times r^2 \\
		u_4 &= u_3 \times r = u_1 \times r^3 \\
		&\vdots \\
		u_n &= u_1 \times r^{n-1}
	\end{align}
	This proof is quite similar to that of the arithmetic sequence's general formula.
\end{proof}

As with arithmetic sequences, solving problems relating to geometric sequences
requires you to identify the first term and the common ratio. From there, you can 
plug values into the general formula to find the desired term.

\section{Geometric Series}

\begin{definition}[Geometric Series]
	A \textit{geometric series} is the sum of the terms of a geometric sequence. If we take the first $n$ terms of a geometric sequence
	and add them together, we get a finite geometric series. 
\end{definition}

With geometric sequences, there is an additional consideration. The sums of infinite geometric 
series are not necessarily undefined, as with arithmetic series. 

\begin{definition}[Convergence]
	A series is said to \textit{converge} if the sum approaches a finite value as more terms are added. 
	If the sum does not approach a finite value, the series is said to \textit{diverge}. To determine 
	whether a geometric series converges or diverges, we look at the common ratio, $r$. If $|r| < 1$, the series converges; 
	otherwise, it diverges.
\end{definition}

\begin{intuition}
	To understand why a geometric series converges when $|r| < 1$, consider the behavior of the terms in the series. 
	When the common ratio $r$ is between -1 and 1, each successive term in the geometric sequence becomes smaller in magnitude. 
	As more terms are added to the series, the contributions of these smaller terms become negligible, leading the sum to approach a finite limit.
	On the other hand, if $|r| \geq 1$, the terms do not decrease in magnitude, and the sum continues to grow without bound, resulting in divergence.
\end{intuition}

We will consider the above in more detail when we consider the infinite sum of a geometric
series. First, let's start with finite sums:

\begin{proposition}
	The sum of the first $n$ terms of a geometric series can be calculated using the formula:
	\begin{equation}
		S_n = u_1 \frac{1 - r^n}{1 - r} \quad \text{for } r \neq 1
	\end{equation}
	where $u_1$ is the first term and $r$ is the common ratio.
\end{proposition}
\begin{proof}
	Let the first $n$ terms of a geometric series be denoted by $S_n$:
	\begin{align}
		S_n &= u_1 + u_2 + u_3 + \ldots + u_n \\
		&= u_1 + u_1 r + u_1 r^2 + \ldots + u_1 r^{n-1}\label{eq:proof-3-1}
	\end{align}
	Multiplying both sides by $r$ gives:
	\begin{equation}\label{eq:proof-3-2}
		r S_n = u_1 r + u_1 r^2 + u_1 r^3 + \ldots + u_1 r^n
	\end{equation}
	Subtracting equation \eqref{eq:proof-3-2} from equation \eqref{eq:proof-3-1}, we get:
	\begin{align*}
		S_n - r S_n &= u_1 - u_1 r^n \\
		S_n (1 - r) &= u_1 (1 - r^n) \\
		S_n &= u_1 \frac{1 - r^n}{1 - r}
	\end{align*}
\end{proof}

The formula is valid for all geometric series. We employ a similar proof for the sum 
of an infinite geometric series, which only applies when the series converges.

\begin{proposition}
	The sum of an infinite geometric series converges to:
	\begin{equation}
		S_\infty = \frac{u_1}{1 - r} \quad \text{for } |r| < 1
	\end{equation}
	where $u_1$ is the first term and $r$ is the common ratio.
\end{proposition}
\begin{proof}
	Let's start at the same place as we did for the finite sum, by writing out th complete sum:
	\begin{equation}\label{eq:proof-3-3}
		S_\infty = u_1 + u_1 r + u_1 r^2 + u_1 r^3 + \ldots
	\end{equation}
	Multiplying both sides by $r$ gives:
	\begin{equation}\label{eq:proof-3-4}
		r S_\infty = u_1 r + u_1 r^2 + u_1 r^3 + \ldots
	\end{equation}
	Subtracting equation \eqref{eq:proof-3-4} from equation \eqref{eq:proof-3-3}, we get:
	\begin{align*}
		S_\infty - r S_\infty &= u_1 \\
		S_\infty (1 - r) &= u_1 \\
		S_\infty &= \frac{u_1}{1 - r}
	\end{align*}
	However, this only holds true if the series converges, which occurs when $|r| < 1$.
\end{proof}