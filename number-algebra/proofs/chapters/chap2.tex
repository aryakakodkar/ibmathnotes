\chapter{Proofs by Counterexample and Contradiction}
\section{Proof by Counterexample}

\begin{definition}[Proof by Counterexample]
    A \textit{proof by counterexample} is a method of disproving a mathematical statement by providing a specific example
    that contradicts the statement.
\end{definition}

When proving a statement to be true, we must show that it holds for all possible
cases. When proving it to be false, we need only to find a single case where the statement
does not hold. For example, the statement "All prime numbers are odd" can be disproven
by providing a single counterexample: the number 2, which is prime but not odd. Proofs by
counterexample are most useful when you are asked to disprove a statement (they are not particularly
useful for proving statements to be true).

\section{Proof by Contradiction}

\begin{definition}[Proof by Contradiction]\label{def:proof-by-contradiction}
    A \textit{proof by contradiction} is a method of proving a mathematical statement by assuming the negation
    of the statement is true, and showing that this assumption leads to a logical contradiction.
\end{definition}

\hyperref[def:proof-by-contradiction]{The definition of a proof by contradiction} is rather abstract. 
In the case of a \hyperref[def:direct-proof]{direct proof}, we assume the premises are true and 
use logical reasoning to arrive at the conclusion. In a proof by contradition, we still assume that the premises
are true, but that the conclusion is false. We then use logical reasoning to arrive at a contradiction (we will
arrive at a statement which contradicts the premises which we assumed to be true).

\begin{proposition}
    $\sqrt{2}$ is irrational.
\end{proposition}
\begin{proof}
    We first remind ourselves of the defintion of a rational number (and indirectly, an irrational number):
    \begin{definition}[Rational Number]\label{def:rational-number}
        A \textit{rational number} is a number that can be expressed as the quotient or fraction $\frac{p}{q}$ of two integers,
        where $p$ and $q$ are integers and $q \neq 0$. An \textit{irrational number} is a number that cannot be expressed as such a quotient.
    \end{definition}
    An \hyperref[def:rational-number]{irrational number} is a number that cannot be expressed as such a quotient. To prove that $\sqrt{2}$ is irrational,
    we will use a proof by contradiction. We will assume that $\sqrt{2}$ is rational, and show that this assumption leads to a contradiction.

    If $\sqrt{2}$ is rational, then we can express it as:
    \begin{equation*}
        \sqrt{2} = \frac{p}{q}
    \end{equation*}
    where $p$ and $q$ are integers with no common factors (i.e., the fraction is in simplest form), and $q \neq 0$.
    Squaring both sides, we get:
    \begin{equation*}
        2 = \frac{p^2}{q^2}
    \end{equation*}
    Multiplying both sides by $q^2$, we have:
    \begin{equation*}
        2q^2 = p^2
    \end{equation*}
    This implies that $p^2$ is even (since it is equal to $2q^2$, which is a multiple of 2). Therefore, $p$ must also be even
    (because the square of an odd number is odd, as you can hopefully prove yourself). We can express $p$ as:
    \begin{equation*}
        p = 2k
    \end{equation*}
    for some integer $k$. Substituting this back into the equation $2q^2 = p^2$, we get:
    \begin{equation*}
        2q^2 = (2k)^2
    \end{equation*}
    Simplifying, we have:
    \begin{equation*}
        2q^2 = 4k^2
    \end{equation*}
    Dividing both sides by 2, we get:
    \begin{equation*}
        q^2 = 2k^2
    \end{equation*}
    This implies that $q^2$ is even, and therefore $q$ must also be even. However, this contradicts our initial assumption that $p$ and $q$ have
    no common factors (since both are even, they share a factor of 2). Therefore, we conclude that the assumptions we made at the start 
    of this proof do not apply: $\sqrt{2}$ is not rational (i.e. it is irrational).
\end{proof}

Since proofs by contradiction are often less intuitive than direct proofs, I will consider a few more examples
of proofs by contradiction. Note that it is somewhat rare to be asked to prove something by contradiction in 
the IB. 

\begin{exercise}[Medium]
    Prove by contradiction that if the integer $n$ is odd, then $n^2$ is also odd.
\end{exercise}
\begin{answer}
    Again, to perform this proof, we must first assume the opposite of the conclusion provided 
    in the statement. In this case, the conclusion is: $n^2$ is odd. Therefore, we will assume that $n^2$ is even. 
    \newline\newline
    We assume that $n$ is odd, and that $n^2$ is even. Since $n^2$ is even, we can express it as:
    \begin{equation*}
        n^2 = 2k \implies n \times n = 2k
    \end{equation*}
    for some integer $k$. 
    But we know that the product of two odd numbers is necessarily odd (we will skip the rigorous proof, but you are 
    encouraged to try to produce it yourself). Hence, we arrive at a contradiction. Therefore, we must conclude that 
    our assumption that $n^2$ is even is false. Thus, if $n$ is odd, then $n^2$ is also odd. 
    
    We could equivalently
    conclude that our assumption that $n$ is odd is the incorrect one, and in doing so, we automatically prove that
    if $n^2$ is even, then $n$ is also even. This is called the \textit{contrapositive} of the original statement. 
    It has the equivalent logical implication as the original statement. This fact is non-examinable.
\end{answer}

\begin{exercise}[Hard]
    Prove that there is no $x \in \mathbb{R}$ such that $\frac{1}{x-2} = 1-x$.
\end{exercise}
\begin{answer}
    A quick reminder: $\mathbb{R}$ is the set of all real numbers. To prove this statement by contradiction, we will assume that
    there exists some $x \in \mathbb{R}$ such that $\frac{1}{x-2} = 1-x$. We will then show that this assumption leads to a contradiction.
    \newline\newline
    Starting from our assumption, we have:
    \begin{equation*}
        \frac{1}{x-2} = 1-x
    \end{equation*}
    Multiplying both sides by $x-2$, we get:
    \begin{equation*}
        1 = (1-x)(x-2)
    \end{equation*}
    Expanding the right-hand side, we have:
    \begin{equation*}
        1 = x - 2 - x^2 + 2x
    \end{equation*}
    Simplifying, we get:
    \begin{equation*}
        1 = -x^2 + 3x - 2
    \end{equation*}
    Rearranging, we have:
    \begin{equation*}
        x^2 - 3x + 3 = 0
    \end{equation*}
    To determine whether this quadratic equation has any real solutions, we can calculate the discriminant:
    \begin{equation*}
        D = b^2 - 4ac = (-3)^2 - 4(1)(3) = 9 - 12 = -3
    \end{equation*}
    Since the discriminant is negative ($D < 0$), the quadratic equation has no real solutions. This contradicts 
    our initial assumption that there exists some $x \in \mathbb{R}$ such that $\frac{1}{x-2} = 1-x$. 
    Therefore, we conclude that there is no $x \in \mathbb{R}$ such that $\frac{1}{x-2} = 1-x$.
\end{answer}