\chapter{Introduction}

\section{Proofs}

In Mathematics, it is usually not sufficient to simply state that some is true; We must 
also provide a \textit{proof} that the statement is indeed true. A proof is a logical argument that demonstrates
the truth of a mathematical statement, based on previously established statements. A number of different kinds of proofs 
exist, each with its own nuances. We will begin with the simplest kind:

\begin{definition}[Direct Proof]
  A \textit{direct proof} is a method of proving a mathematical statement by assuming the premises are true
  and using logical reasoning to arrive at the conclusion.
\end{definition}

When I say \textit{premises}, I mean the initial assumptions or conditions of the statement.
For those confused, an example may clear things up:

\begin{eg}
  Prove that if $n$ is an even integer, then $n^2$ is also even.
\end{eg}

Let's break down the statement in the above example to determine what the premises are,
and what the conclusion of our proof must be. The statement begins with "if \( n \) is an even integer", which is our premise.
We assume that $n$ is indeed an even integer, because we don't need to consider the alternative case (that $n$ is odd),
since the statement in the example doesn't concern it.

Our conclusion will be that $n^2$ is even, if $n$ is even. For a direct proof, we must show this to be true
using logical reasoning. Enough talk, let's get to the proof itself:
\begin{proposition}
  If $n$ is an even integer, then $n^2$ is also even.
\end{proposition}
\begin{proof}
  Let $n$ be an even integer. By definition, an even integer is a multiple of 2. 
  Therefore, we can express $n$ as:
  \begin{equation*}
    n = 2k
  \end{equation*}
  for any integer $k$ (e.g. if $n=8$, then $k=4$ --- this can be done for any even $n$). 
  Now, we will compute $n^2$:
  \begin{align*}
    n^2 &= (2k)^2 \\
        &= 4k^2 \\
        &= 2(2k^2)
  \end{align*}
  We clearly see that $n^2$ is a multiple of 2 (since $2k^2$ is an integer). Therefore, by definition, $n^2$ is even.
\end{proof}

In the above proof, we have mathematically re-written our premise ($n \text{ even} \rightarrow n = 2k$),
and used it to logically arrive at our conclusion. It is common to be uncomfortable with how I came up with 
the first line of this proof (rewriting $n$ as $2k$). This is a skill that comes with practice, and simply being
exposed to many proofs. Let's look at a few more examples of direct proofs, and the most common examples of premises.

\begin{exercise}[Easy]
  Prove that the sum of any 3 consecutive integers is divisible by 3.
\end{exercise}
\begin{answer}
  Let us begin by defining our premise. It is possible to rewrite this statement as: if 3 integers are consecutive, 
  prove that their sum is divisible by 3. Therefore, our premise is that we have 3 consecutive integers.
  We can write this by considering the first integer to be some integer $n$. Then, the next two consecutive integers are $n+1$ 
  and $n+2$. The sum of these integers is:
  \begin{align*}
    \text{Sum} &= n + (n+1) + (n+2) \\
               &= 3n + 3 \\
               &= 3(n + 1)
  \end{align*}
  Since $n + 1$ is an integer, we see that the sum of the 3 consecutive integers is a multiple of 3.
  Therefore, the sum is divisible by 3.
\end{answer}

\begin{exercise}[Easy]
  Prove that $x^2 - 3x + 3$ is always positive for all real values of $x$. 
\end{exercise}
\begin{answer}
  Again, it may help to re-write the statement in our `if-then' format. This one becomes: 
  if $x$ is a real number, prove that $x^2 - 3x + 3 > 0$. Therefore, our premise is that $x$ is a real number.
  
  \begin{figure}[H]
    \centering
    \begin{tikzpicture}
      \begin{axis}[
          axis lines = middle,
          xlabel = $x$,
          ylabel = {$f(x)$},
          domain=-1:4,
          samples=100,
          ymin=-1, ymax=5,
          xtick=\empty,
          ytick=\empty,
          grid=none
        ]
        \addplot [red, thick] {x^2 - 3*x + 3};
      \end{axis}
    \end{tikzpicture}
    \caption{The graph of the quadratic function $f(x) = x^2 - 3x + 3$.}
  \end{figure}
  For those familiar with quadratics (if you aren't, see my notes on functions), we note that this function is concave up. 
  For it to be positive at all real $x$, it cannot cross the x-axis (if it does, the value of $f(x)$ reaches $0$, which is not positive). 
  Equivalently, the function must have no real roots. Its discriminant must therefore be less than 0:
  \begin{align*}
    D &= b^2 - 4ac \\
      &= (-3)^2 - 4(1)(3) \\
      &= 9 - 12 \\
      &= -3
  \end{align*}
  Since $D < 0$, the quadratic has no real roots, and is therefore always positive for all real values of $x$.
\end{answer}