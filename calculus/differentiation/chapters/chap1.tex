\chapter{Limits and Continuity}

\section{Limits}\label{sec:limits}

Limits are a fundamental concept in calculus that describe the behavior of a function as its input 
approaches a certain value. We will not consider the rigorous definition of a limit here, but 
rather focus on the intuititive understanding of limits and how to compute them.

\begin{definition}[Limit of a Function]
    The \textit{limit} of a function $f(x)$ as $x$ approaches a value $c$ is the value that $f(x)$ approaches
    as $x$ gets closer and closer to $c$. This is denoted as:
    \begin{equation*}
        \lim_{x \to c} f(x) = L
    \end{equation*}
    where $L$ is the value that $f(x)$ approaches as $x$ approaches $c$.
\end{definition}

Now, this is a rather abstract definition. To understand it better, let us consider an example:
\begin{eg}
    Consider the function $f(x) = \frac{x^2 - 1}{x - 1}$. We want to find the limit of $f(x)$ as $x$ approaches 1:
    \begin{equation*}
        \lim_{x \to 1} \frac{x^2 - 1}{x - 1}
    \end{equation*}
    If we directly substitute $x = 1$ into the function, we get:
    \begin{equation*}
        f(1) = \frac{1^2 - 1}{1 - 1} = \frac{0}{0}
    \end{equation*}
    which is undefined. Instead, let's consider what happens to the function when it is very close to 1.
    Let's consider $x=1.1$:
    \begin{equation*}
        f(1.1) = \frac{(1.1)^2 - 1}{1.1 - 1} = \frac{0.21}{0.1} = 2.1
    \end{equation*}
    What if we got even closer? Let's try $x=1.01$:
    \begin{equation*}
        f(1.01) = \frac{(1.01)^2 - 1}{1.01 - 1} = \frac{0.0201}{0.01} = 2.01
    \end{equation*}
    And even closer, $x=1.001$:
    \begin{equation*}
        f(1.001) = \frac{(1.001)^2 - 1}{1.001 - 1} = \frac{0.002001}{0.001} = 2.001
    \end{equation*}
    We can see a pattern here: as $x$ gets closer and closer to 1, $f(x)$ gets closer and closer to 2. We conclude that 
    \begin{equation*}
        \lim_{x \to 1} \frac{x^2 - 1}{x - 1} = 2
    \end{equation*}
    Out loud, we would say: "The limit of $f(x)$ as $x$ approaches 1 is 2."
    We have considered values of $x$ that are slightly greater than 1 (i.e. we approach 1 from the right on a graph). 
    Try considering values of $x$ that are slightly less than 1 (i.e. approach 1 from the left on a graph) to verify that the limit is indeed 2.
\end{eg}

\begin{definition}[Existence of a Limit]
    We say that the limit of a function $f(x)$ as $x$ approaches $c$ \textit{exists} if the left-hand limit
    and right-hand limit are equal. That is:
    \begin{equation*}
        \lim_{x \to c^-} f(x) = \lim_{x \to c^+} f(x) = L
    \end{equation*}
    where $L$ is the value that $f(x)$ approaches as $x$ approaches $c$. In this case, we write:
    \begin{equation*}
        \lim_{x \to c} f(x) = L
    \end{equation*}
\end{definition}

\begin{notation}
    The notation $\lim_{x \to c^-} f(x)$ denotes the \textit{left-hand limit} of $f(x)$ as $x$ approaches $c$ from values less than $c$.
    The notation $\lim_{x \to c^+} f(x)$ denotes the \textit{right-hand limit} of $f(x)$ as $x$ approaches $c$ from values greater than $c$.
\end{notation}

\begin{eg}[Non-existent Limit]
    Consider the function $f(x)$ defined as:
    \begin{equation*}
        f(x) = 
        \begin{cases} 
            2 & \text{if } x < 1 \\
            3 & \text{if } x \geq 1 
        \end{cases}
    \end{equation*}
    We want to find the limit of $f(x)$ as $x$ approaches 1:
    \begin{equation*}
        \lim_{x \to 1} f(x)
    \end{equation*}
    Let's consider the left-hand limit:
    \begin{equation*}
        \lim_{x \to 1^-} f(x) = 2
    \end{equation*}
    Now, let's consider the right-hand limit:
    \begin{equation*}
        \lim_{x \to 1^+} f(x) = 3
    \end{equation*}
    Since the left-hand limit (2) and right-hand limit (3) are not equal, we conclude that the limit of $f(x)$ as $x$ approaches 1 does not exist.
    Perhaps a graphical representation of this function would help illustrate this concept better.
    \begin{figure}[H]
    \centering
      \begin{tikzpicture}
          \begin{axis}[
              axis lines = middle,
              xlabel = $x$,
              ylabel = {$f(x)$},
              ymin=0, ymax=4,
              xmin=0, xmax=2,
              xtick={1},
              ytick={2,3},
              yticklabels={2,3},
              domain=0:2,
              samples=100,
              width=10cm,
              height=6cm,
          ]
              % Left piece: y=2 for x in [0, just before 1)
              \addplot[blue, thick, domain=0:0.98, samples=2] {2};
              % Open circle at (1,2) with blue stroke, no fill
              \addplot[blue, only marks, mark=o, mark options={scale=1.2,draw=blue,fill=white,line width=1.2pt}] coordinates {(1,2)};
              % Right piece: y=3 for x in [1,2]
              \addplot[red, thick, domain=1:2, samples=2] {3};
              % Solid dot at (1,3)
              \addplot[red, only marks, mark=*, mark options={scale=1.2}] coordinates {(1,3)};
          \end{axis}
      \end{tikzpicture}
      \caption{Graph of the piecewise function $f(x)$}
  \end{figure}

Clearly, from the left side (blue), the function approaches 2, while from the right side (red), 
the function approaches 3. This limit does not exist. Note that we have used an open circle on the left side at $x=1$ to indicate that the function 
does not include that poin (i.e. $f(x) = 2$ when $x<1$, but not when $x=1$).
\end{eg}

\section{Continuity}

Continuity of a function is a fairly intuitive concept. A discontinuous function is
one which contains a `jump' or `break' in its graph. Keep this intuition in mind as 
we define continuity more formally.

\begin{definition}[Continuity at a Point]\label{def:continuity}
    A function $f(x)$ is said to be \textit{continuous} at a point $x = c$ if the following three conditions are met:
    \begin{enumerate}
        \item $f(c)$ is defined (i.e. $c$ is in the domain of $f$, and not $f(c) \rightarrow \infty$).
        \item The limit of $f(x)$ as $x$ approaches $c$ exists.
        \item The limit of $f(x)$ as $x$ approaches $c$ is equal to $f(c)$:
        \begin{equation*}
            \lim_{x \to c} f(x) = f(c)
        \end{equation*}
    \end{enumerate}
\end{definition}

\begin{definition}[Continuity on an Interval]
    A function $f(x)$ is said to be \textit{continuous} on an interval $a \le x \le b$ if it is continuous at every point $x$ in the interval (i.e.
    every point between $a$ and $b$, inclusive).
\end{definition}

\begin{definition}[Continuous Function]
    A function $f(x)$ is said to be a \textit{continuous function} if it is continuous at every point in its domain.
    A function which is not continuous is called \textit{discontinuous}.
\end{definition}

While the \hyperref[def:continuity]{rigorous definition of continuity} is somewhat complex, deciding whether a function
is continuous is often just a matter of determining whether there are any `jumps' or `breaks' in its graph. A function also 
becomes discontinuous if there is a vertical asymptote (i.e. the function approaches infinity at some point). These 
are the phenomena we look out for when determining whether a function is continuous.

\begin{exercise}[Easy]
    Determine whether the function $f(x)$ defined as:
    \begin{equation}\label{eq:continuity-ex1}
        f(x) = 
        \begin{cases} 
            x^2 & \text{if } x \neq 2 \\
            5 & \text{if } x = 2 
        \end{cases}
    \end{equation}
    is continuous at $x=2$.
\end{exercise}
\begin{answer}
    To determine whether $f(x)$ is continuous at $x=2$, we will check the three conditions from the definition of continuity:
    \newline\newline
    \textbf{Condition 1:} $f(2)$ is defined. From the definition of $f(x)$, we see that $f(2) = 5$. Therefore, condition 1 is satisfied.
    \newline\newline
    \textbf{Condition 2:} The limit of $f(x)$ as $x$ approaches 2 exists. We need to compute:
    \begin{equation*}
        \lim_{x \to 2} f(x)
    \end{equation*}
    Since $f(x) = x^2$ for all $x \neq 2$, we can compute the limit using this expression:
    \begin{equation*}
        \lim_{x \to 2} x^2 = 2^2 = 4
    \end{equation*}
    Any polynomial function is continuous everywhere, so we can compute the limit by direct substitution. 
    Therefore, condition 2 is satisfied.
    \newline\newline
    \textbf{Condition 3:} The limit of $f(x)$ as $x$ approaches 2 is equal to $f(2)$. We have:
    \begin{equation*}
        \lim_{x \to 2} f(x) = 4
    \end{equation*}
    and
    \begin{equation*}
        f(2) = 5
    \end{equation*}
    Since $4 \neq 5$, condition 3 is not satisfied.
    \newline\newline
    Since condition 3 is not satisfied, we conclude that the function $f(x)$ is not continuous at $x=2$.
    \begin{figure}[H]
    \centering
    \begin{tikzpicture}
      \begin{axis}[
          axis lines = middle,
          xlabel = $x$,
          ylabel = {$f(x)$},
          domain=0:4,
          samples=100,
          ymin=0, ymax=8,
          xtick=\empty,
          ytick=\empty,
          ytick={4,5},
          yticklabels={$4$,$5$},
          xtick={2},
          grid=none,
          width=10cm,
          height=6cm
        ]
        % Plot y = x^2 except at x=2
        \addplot [red, thick, domain=0:1.98] {x^2};
        \addplot [red, thick, domain=2.02:4] {x^2};
        % Open circle at (2,4)
        \addplot [red, only marks, mark=o, mark options={scale=1.2,draw=red,fill=white,line width=1.2pt}] coordinates {(2,4)};
        % Solid dot at (2,5)
        \addplot [red, only marks, mark=*, mark options={scale=1.2}] coordinates {(2,5)};
      \end{axis}
    \end{tikzpicture}
    \caption{Graph of equation \ref{eq:continuity-ex1}}
\end{figure}
\end{answer}