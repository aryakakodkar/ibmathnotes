\chapter{Proofs by Counterexample and Contradiction}
\section{Proof by Counterexample}

\begin{definition}[Proof by Counterexample]
    A \textit{proof by counterexample} is a method of disproving a mathematical statement by providing a specific example
    that contradicts the statement.
\end{definition}

When proving a statement to be true, we must show that it holds for all possible
cases. When proving it to be false, we need only to find a single case where the statement
does not hold. For example, the statement "All prime numbers are odd" can be disproven
by providing a single counterexample: the number 2, which is prime but not odd. Proofs by
counterexample are most useful when you are asked to disprove a statement (they are not particularly
useful for proving statements to be true).

\section{Proof by Contradiction}

\begin{definition}[Proof by Contradiction]\label{def:proof-by-contradiction}
    A \textit{proof by contradiction} is a method of proving a mathematical statement by assuming the negation
    of the statement is true, and showing that this assumption leads to a logical contradiction.
\end{definition}

\hyperref[def:proof-by-contradiction]{The definition of a proof by contradiction} is rather abstract. 
In the case of a \hyperref[def:direct-proof]{direct proof}, we assume the premises are true and 
use logical reasoning to arrive at the conclusion. In a proof by contradition, we still assume that the premises
are true, but that the conclusion is false. We then use logical reasoning to arrive at a contradiction (we will
arrive at a statement which contradicts the premises which we assumed to be true).

\begin{proposition}
    $\sqrt{2}$ is irrational.
\end{proposition}
\begin{proof}
    We first remind ourselves of the defintion of a rational number (and indirectly, an irrational number):
    \begin{definition}[Rational Number]\label{def:rational-number}
        A \textit{rational number} is a number that can be expressed as the quotient or fraction $\frac{p}{q}$ of two integers,
        where $p$ and $q$ are integers and $q \neq 0$. An \textit{irrational number} is a number that cannot be expressed as such a quotient.
    \end{definition}
    An \hyperref[def:rational-number]{irrational number} is a number that cannot be expressed as such a quotient. To prove that $\sqrt{2}$ is irrational,
    we will use a proof by contradiction. We will assume that $\sqrt{2}$ is rational, and show that this assumption leads to a contradiction.

    If $\sqrt{2}$ is rational, then we can express it as:
    \begin{equation*}
        \sqrt{2} = \frac{p}{q}
    \end{equation*}
    where $p$ and $q$ are integers with no common factors (i.e., the fraction is in simplest form), and $q \neq 0$.
    Squaring both sides, we get:
    \begin{equation*}
        2 = \frac{p^2}{q^2}
    \end{equation*}
    Multiplying both sides by $q^2$, we have:
    \begin{equation*}
        2q^2 = p^2
    \end{equation*}
    This implies that $p^2$ is even (since it is equal to $2q^2$, which is a multiple of 2). Therefore, $p$ must also be even
    (because the square of an odd number is odd, as you can hopefully prove yourself). We can express $p$ as:
    \begin{equation*}
        p = 2k
    \end{equation*}
    for some integer $k$. Substituting this back into the equation $2q^2 = p^2$, we get:
    \begin{equation*}
        2q^2 = (2k)^2
    \end{equation*}
    Simplifying, we have:
    \begin{equation*}
        2q^2 = 4k^2
    \end{equation*}
    Dividing both sides by 2, we get:
    \begin{equation*}
        q^2 = 2k^2
    \end{equation*}
    This implies that $q^2$ is even, and therefore $q$ must also be even. However, this contradicts our initial assumption that $p$ and $q$ have
    no common factors (since both are even, they share a factor of 2). Therefore, we conclude that the assumptions we made at the start 
    of this proof do not apply: $\sqrt{2}$ is not rational (i.e. it is irrational).
\end{proof}